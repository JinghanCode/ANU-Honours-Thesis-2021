\chapter*{Abstract}
\addcontentsline{toc}{chapter}{Abstract}
\vspace{-1em}
Property based testing is a testing approach where tests specify properties for possible test inputs, which are used to generate test cases to show bugs in the system under test. Property based tests find more bugs than traditional unit tests, as traditional unit tests only test a subset of test cases which can be generated by property based tests. However, any testing approach involving concrete execution of test cases is never guaranteed to catch all bugs that possibly exist in a program. Symbolic solver-based testing can find bugs that are missed by concrete property based testing and is a possible solution to this limitation. However, solver-based approaches suffer from path explosion and solvers can fail for complex operations and code. To this end a novel approach for hybrid concrete/symbolic testing is proposed. The approach involves running a test twice with a concrete property based testing tool Hypothesis and a symbolic solver-based tool Crosshair. This was implemented by building support for reading Hypothesis tests into Crosshair. This combined testing approach was validated using various benchmarks and real-world bug hunting with open-source projects. TODO summarise findings.


%%% Local Variables: 
%%% mode: latex
%%% TeX-master: "paper"
%%% End: 
